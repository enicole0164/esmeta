\section{Introduction}\label{sec:intro}

In the beginning, JavaScript was a simple scripting language designed in a
ten-day hack. However, it has become a de facto Web standard and eventually
becomes one of the dominating programming languages in various fields. For
example, Node.js\footnote{https://nodejs.org/} introduced full-stack JavaScript
by supporting server-side programming.  Developers can develop even mobile and
desktop applications for many platforms in JavaScript with React
Native\footnote{https://reactnative.dev/}.  Furthermore, according to the annual
report of GitHub\footnote{\url{https://octoverse.github.com/}}, JavaScript has
consistently been the most popular programming language based on the number of
contributors to GitHub projects.

Developers and researchers have presented diverse tools for JavaScript.  In
industry, developers develop and maintain various JavaScript engines in
different fields, such as Google V8\cite{v8}, Oracle GraalJS\cite{graaljs}, and
Moddable XS\cite{xs}.  Beyond the JavaScript engine, developers actively develop
compilers as well; Babel\cite{babel} compiles modern >ES6 JavaScript code to
ES5.1 code, TypeScript compiler compiles TypeScript code to JavaScript code.  In
academia, researchers have presented various techniques for JavaScript,
including static analysis~\cite{safe, safe2, tajs, wala, jsai}, double
debuggers~\cite{jsexplain}, verification~\cite{javert, javert2, ad-safety,
javanni}, symbolic execution~\cite{symbolic-js, sym-js, expo-se}, and concolic
testing~\cite{jalangi, type-conc-test}.

Therefore, they have striven to understand the detailed semantics of JavaScript
to design their own tools and techniques correctly.

Therefore, 
For the developement of such tools, It is reuiqred 
To understand 

However, it is infeasible to know all the detailed complex semantics by reading
the specification even to language designers. For example, consider a JavaScript
logical nullish assignment
\jscode{x ??= y} newly introduced in the latest ECMAScript (ES12,
2021)~\cite{es12}. Its basic semantics is to assign \jscode{y} to \jscode{x}
only if \jscode{x} is nullish (\jscode{undefined} or \jscode{null}). Therefore,
it seems to have exactly same semantics with an assignment \jscode{x = x ?? y}
with a nullish coalescing operator \jscode{??}.  However, it is not true because
of the short-circuiting for assignments:
\begin{lstlisting}[style=JS]
       const x = 0;  x ??= 1;    // x === 0
       const y = 0;  y = y ?? 1; // TypeError
\end{lstlisting}
and the named evaluation for anonymous functions:
\begin{lstlisting}[style=JS]
   let f;  f ??= () => {};      // f.name === "f"
   let g;  g = g ?? (() => {}); // g.name === ""
\end{lstlisting}
Such misunderstanding of complex semantics in various language features causes
the wrong implementation of JavaScript applications or even JavaScript engines.

% Thus, researchers have presented several tools for a better understanding of
% JavaScript semantics. For example, JSExplain~\cite{jsexplain} is a reference
% interpreter that provides step-by-step execution of a given JavaScript program
% by closely following the English sentences of the specification. Several
% JavaScript language designers accept the concept of reference interpreters and
% implement Narcissus~\cite{narcissus} and engine262~\cite{engine262} to use in
% the language design process. On the other hand, TC39 feels the need for type
% information in ECMAScript to enhance its readability and has started internal
% discussions on manual type annotations for each abstract
% algorithm\footnote{https://github.com/tc39/ecmarkup/issues/173}. To meet the
% demand of type information, \citet{jstar} presents a tool named $\jstar$ to
% perform a type analysis for a given ECMAScript. Besides, they detect
% type-related specification bugs in ES12 using the type analysis result. Finally,
% $\jest$~\cite{jest} is another tool that synthesizes JavaScript conformance
% tests from a given ECMAScript. Thus, it allows users to understand language
% semantics using test programs instead of English sentences in the specification.
% 
% However, existing tools have two limitations: 1) manual update for evolving
% language specification and 2) no user interactions to select target semantics.
% Among existing tools, JavaScript reference interpreters require manual updates
% when ECMAScript evolves. Until 2015, it was not a critical problem because
% ECMAScript had rarely evolved. However, TC39 decided to annually release the
% specification with a massive update in ECMAScript 6 (ES6, 2015)~\cite{es6} in
% 2015. Moreover, they published the specification as an open-source project in a
% GitHub repository to quickly adapt users' demands to the language. For example,
% JSExplain also requires manual updates for new specifications. Thus, it still
% supports only ECMAScript 5.1 (ES5.1, 2011)~\cite{es5}, while already seven more
% versions from ES6 to ES12 have been released. Another limitation is that
% existing tools do not support user interactions to select target semantics to
% get additional information. For example, $\jstar$ and $\jest$ do not interact
% with users to give more information helpful for semantics understanding.
% Instead, they only extract type information and conformance tests for a given
% ECMAScript. On the other hand, while JSExplain supports step-by-step execution
% as user interaction with several debugger-like features, it can deal with only
% the execution trace of a single JavaScript program.
% 
% In this paper, we present $\tool$, a \textbf{J}ava\textbf{S}cript
% \textbf{I}nteractive \textbf{S}pecification. It is the first reference
% interpreter automatically synchronized with a given ECMAScript supporting
% step-by-step program execution. Fortunately, \citet{jiset} presents a tool named
% $\jiset$ that extracts a mechanized specification from a given specification.
% The extracted mechanized specification is written in a specification language
% $\ires$ and executable with a JavaScript program. Thus, we implement our tool
% using $\jiset$ and design debugger-like features: stepping controls, state
% visualization, and breakpoints with algorithm names or JavaScript program
% points. Moreover, our tool supports two different types of user interactions to
% select target semantics: \textit{syntactic view selection} and \textit{algorithm
% step selection}. First, we introduce a way to reduce abstract algorithms using a
% partial evaluation~\cite{peval, peval-survey, trans-ai} with a \textit{syntactic
% view}. A syntactic view is a JavaScript Abstract Syntax Tree (AST) defined with
% abstracted nodes. We formally define a partial evaluation with syntactic views
% and prove its semantics preservation under the given syntactic view. Second, we
% also present a way to extract a small-sized program for each algorithm step
% using delta debugging~\cite{delta-debugging} with a given set of programs. For
% the f first step, we execute each program using $\tool$ to compute semantics
% coverage in ECMAScript and filter out unnecessary programs. Then, we perform
% delta debugging for each program to reduce program sizes without loss of
% semantics coverage.
% 
% Our contributions are as follows:
% \begin{itemize}
%   \item We present $\tool$, a JavaScript interactive specification. It is the
%     first reference interpreter automatically synchronized with a given
%     ECMAScript supporting \textit{step-by-step program execution} with various
%     debugger-like featuress.
% 
%   \item We introduce two different types of user interactions to select target
%     semantics and implement them in $\tool$: 1) a \textit{syntactic view
%     selection} to reduce abstract algorithms using \textit{partial evaluation}
%     and 2) an \textit{algorithm step selection} to extract a small-sized
%     JavaScript program related to the selected step using \textit{delta
%     debugging} with a given set of programs.
% 
%   \item We evaluate $\tool$ with the latest ECMAScript, ES12, and experimentally
%     show the effectiveness of two different target semantics selections with
%     \inred{177} syntactic views and \inred{19,839} programs. Our tool
%     successfully supports step-by-step program execution in ES12. It reduces
%     \inred{18.9}\% algorithm steps for each syntactic view and provides
%     \inred{2.3} lines (\inred{152.3} bytes) of a program for each step on
%     average.
% \end{itemize}
