\section{Introduction}\label{sec:intro}

In the beginning, JavaScript was a simple scripting language designed in a
ten-day hack. However, it has become a de facto Web standard and eventually
became one of the dominating programming languages in various fields. For
example, Node.js~\cite{nodejs} introduced full-stack JavaScript by supporting
server-side programming.  In addition, React Native~\cite{react-native} provides
a way to deploy JavaScript programs as mobile and desktop applications even
running on different platforms.  Furthermore, according to the annual report of
GitHub, JavaScript has consistently been the most popular programming language
based on the number of contributors to GitHub
projects.\footnote{\url{https://octoverse.github.com/}}

However, despite its popularity, developers and even JavaScript experts often
suffer from its highly dynamic nature.  It supports diverse language features
with complex semantics, including higher-order functions, mutable prototype
chains, dynamic code generation, and asynchronous functions.  For instance, even
addition and subtraction operations are not straightforward with complicated
implicit conversion rules.  Consider the following simple program:
\begin{center}
  \begin{tabular}{c}
    \begin{lstlisting}[style=JS]
'42' + 1 - 1 // result: 420
    \end{lstlisting}
  \end{tabular}
\end{center}
Surprisingly, it produces a number $\code{420}$ because the $\code{+}$
operator between string and number values denotes a string concatenation while
the $\code{-}$ operator between them denotes a numerical subtraction.  Such a
highly dynamic nature of JavaScript causes pain even to JavaScript experts in
understanding the language semantics.  For example, while V8 is the most
widely-used JavaScript engine developed by Google for Node.js and Chrome web
browsers, its semantic bugs have consistently been reported to its issue
tracker.\footnote{\url{https://bugs.chromium.org/p/v8}}

Therefore, when they desire to understand the detailed semantics of JavaScript,
they refer to ECMA-262, the standard JavaScript language specification written
in English.  It describes the syntax using a variant of the extended Backus–Naur
form and the semantics as abstract algorithms.  Each algorithm describes the
language semantics in with ordered and structured steps.  The Ecma Technical
Committee 39 (TC39) has officially maintained ECMA-262 as an open-source GitHub
project,\footnote{\url{https://github.com/tc39/ecma262}} and it annually updates
the specification with new language features to quickly adapt to the users'
demands.  Thus, JavaScript developers should read this specification if they
want to learn the detailed semantics of new language features introduced every
year.  Especially, developers of JavaScript tools (e.g.,
interpreters~\cite{v8, jscore, spidermonkey, graaljs},
transpilers~\cite{babel},
code analysis tools~\cite{safe, safe2, tajs, wala, jsai},
verification tools~\cite{javert, javert2, ad-safety, javanni},
symbolic execution engines~\cite{symbolic-js, sym-js, expo-se},
or concolic testing engines~\cite{jalangi, type-conc-test})
must carefully read the detailed semantics of all language features described in
the JavaScript language specification.

Unfortunately, it is still challenging to 1) understand all the \textit{possible
behaviors} of language features and 2) find \textit{example programs} that
trigger such behaviors by only referring to ECMA-262.

First, it is not an easy task to answer which behaviors are possible in each
language feature.  It often provides hidden implicit behaviors unusual in other
languages, such as implicit function calls, implicit property reads/writes, or
runtime exceptions under special conditions.  For example, we can call functions
even using the addition operation without any function application syntax.  If
we want to understand all possible behaviors of the addition operation, we need
to read \inred{109} algorithms reachable from its evaluation algorithm.
Besides, the problem becomes more complicated if we want to know possible
behaviors of a language feature used with specific types of values or other
language features.

Second, even if users realize its specific behavior, it is nontrivial to find
example programs that trigger the behavior.  For example, assume that we want to
find an example program having an addition operation that causes implicit
function calls.  Then, we should first grasp the condition when the addition
operation implicitly invokes functions and then find a JavaScript program that
satisfies the condition like this:
\begin{center}
  \begin{tabular}{c}
    \begin{lstlisting}[style=JS]
"a" + {toString: ()=>"b"} // result: "ab"
    \end{lstlisting}
  \end{tabular}
\end{center}
In this code, the left and right operands have different types $\code{String}$
and $\code{Object}$, respectively.  The addition between them causes an
implicit conversion of the object to a string value, and the function in the
$\code{toString}$ property is implicitly invoked during the conversion.

To alleviate this problem, this paper presents $\tool$ for the advanced
understanding of JavaScript language specification with \textit{syntactic
views}.  We introduce a \textit{syntactic view}, an extension of a JavaScript
abstract syntax tree (AST) consisting of both concrete and abstract nodes with
type bounds.  Users can freely construct syntactic views depending on
which combination of language features they want to understand; concrete nodes
denote the parts they want to understand, and abstract ones denote the parts
they do not.

For a given syntactic view, the tool performs reachability analysis to
automatically reduce the call graph of ECMA-262 with a given syntactic view to
soundly infer \textit{possible behaviors}.  Our key observation is that the
reachability of specific algorithms in the language specification implies
possible behaviors of language features.  For example, the \esalg{Call}
algorithm in ECMA-262 represents the invocation of functions.  Thus, if the
reduced call graph of ECMA-262 with a syntactic view for the addition operation
contains the \esalg{Call} algorithm, it might cause implicit function calls.

In addition, $\tool$ searches for \textit{example programs} that trigger
possible behaviors in a given program pool.  The tool first collects execution
traces of all the programs in the pool and builds a database only once at
initialization time.  Then, if a user selects an algorithm as the target
behavior with a syntactic view, $\tool$ utilizes the database to quickly search for
example programs in the pool that touches the selected algorithm in the
semantics of the given syntactic view.  For example, if we selects the
\esalg{Call} algorithm with a syntactic view for addition operation, the tool
automatically searches for example programs having an addition operation that
causes implicit function calls.

Our contributions are as follows:
\begin{itemize}
  \item We introduce a \textit{syntactic view}, an extension of a JavaScript
    abstract syntax tree (AST) consisting of both concrete and abstract nodes
    with type bounds, to help users indicate which language features
    they want to understand.

  \item We present an automatic way to 1) infer \textit{possible behaviors} of
    JavaScript language features using a reachability analysis in ECMA-262 and
    2) search \textit{example programs} that trigger such behaviors in a given
    program pool using their pre-collected execution traces.

  \item We actualize our techniques in $\tool$, evaluate its effectiveness using
    the latest ECMA-262 with \inred{177} basic syntactic views, and demonstrate
    its practicality with \inred{three} advanced syntactic views.
\end{itemize}
